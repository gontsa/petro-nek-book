\section*{Курт Воннеґут: Як написати коротке оповідання}
\addcontentsline{toc}{section}{Курт Воннеґут: Як написати коротке оповідання}

Я тут вирішив що час від часу буду викладати тут переклади. Мені мовна практика, а вам може щось корисне знайдеться. Переклади не ідеальні, проте I’ll do my best. Якщо знаходите нестиковки в перекладі, прошу виправляти. Сьогодні \textbf{вісім порад як написати оповідання від Курта Воннеґута}:

\begin{enumerate}

 \item Використовуйте час людей котрих ви взагалі не знаєте так щоб вони не  відчули що змарнували його.
 \item Дайте читачу хоча б одного персонажа за котрого б він вболівав.
 \item Кожен персонаж має чогось хотіти, навіть якщо це лише склянка води.
 \item В кожному реченні має бути одне з двох — розкриття персонажу чи розвиток подій.
 \item Почніть настільки ближче до кінця наскільки це можливо.
 \item Будьте садистом. \textit{Не має значення якими милими та невинними є ваші герої, робіть так, щоб з ними ставалися жахливі речі, так щоб читач побачив з чого вони зроблені.}
 \item Пишіть щоб сподобатись лише одній людині. \textit{Якщо ви відкриєте вікно та догоджатимете цілому світу, тоді ваше оповідання застудиться та захворіє.}
 \item Дайте читачам настільки багато інформації наскільки це важливо і чим швидше тим краще. \textit{До біса очікування. Читачі повинні мати повне розуміння того що відбувається, де і чому, так щоб вони могли закінчити розповідь самими, якщо б таракани з’їли останні кілька сторінок.}
\end{enumerate}