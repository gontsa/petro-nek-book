\section*{Патетична Соната}
\addcontentsline{toc}{section}{Патетична Соната}

Слухаю Бетховена і читаю Куліша, того що Микола. Отримую море задоволення.Ось пару цитат:
\begin{flushright}

\begin{verse}
\textbf{Другий} (до першого). Ви бачили такого малахольного?

\textit{Перший}. Чого він хоче?

\emph{Другий}. Він хоче, щоб йому вже нація ботинки чистила.

Варваро Михайлівно! Здрастуйте, милая! Друга. Не здрастуйте, а Христос воскрес! Перша. Аж тепер я зрозуміла, як він, бідненький, зрадів, коли воскрес!

Оврам. Раз ноги не мають значення, то й питання не має значення.

Голова. Ви хочете скласти провину на ноги!

Оврам. Навіщо, коли винні ваші голови.


Пероцький. Кошмар! Уночі до мене в камеру всадили монаха, і він цілу ніч молився. По-вкраїнському. Ви розумієте, панове! Він не давав мені спати.

Пероцький. Кошмар! Уночі до мене в камеру всадили монаха, і він цілу ніч молився. По-вкраїнському. Ви розумієте, панове! Він не давав мені спати.

Простіть, що не вийшла назустріч, не одчинила дверей (жест нагору) в мою країну, але, як бачите, провина не моя. Мою країну в мене одібрали.

Скажіть, у що лучче мені вдягтися?
Я. У щирість і мужність…

Я вийду з хати безсонна од бажання – дівка з відрами. Ви біля криниці. Поведу вас у жито… (Вона, немов справді жито, розгортає руками і веде мене уявного). А в житі волошки голубіють, стелеться біленька березка. Бачили? Боже! Як пахне любов!
\end{verse}
\end{flushright}
