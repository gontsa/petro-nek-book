\section*{«Почему полувагоны грузите? Грузите полные!»}
\addcontentsline{toc}{section}{«Почему полувагоны грузите? Грузите полные!»}



Корисно так деколи поспілкуватися з людьми різних професій. Дізнаєшся багато цікавих історій. Недавно спілкувався з залізничниками. Ділюся мудростями та бувальщинами.

От, певно всі бачили, приїжджає поїзд, а коло нього ходить дядько і стукає по кришках коліс молотком. От навіщо він це робить? Я колись думав  що це пеервірка чи все добре прикручено, все добре тримається чи можливо рівень мастила, а ні насправді стукають для того щоб перевірити температуру колісних пар. Коли вони гарячі, то звук буде іншим ніж коли вони холодні. А гарячими можуть бути тоді, коли клинить підшибники, колісні пари не крутяться і від тертя нагріваються. В таких випалках можливий схід вагонів. Тому от ходять і таким чином перевіряють температуру коліс. А ще в залізничників на перегонах стоїть система панап (якщо я правильно запам’ятав), котра оптичним методом автоматично визначає температуру колісних пар, і якщо десь є перевищення норми, то посилається сигнал в тепловоз і йде процес тормозіння де вказується вагон та колісна пара що була гаряча для перевірки вручну. Якогось разу доповідали керівництву причину зупинки рухомого складу. Сказали що поїзд зупинив панап, на що отримали відповідь: «Панапа уволить, остальным выговор».

А ще на залізниці є різні скорочення, от там черговий по станції чи хтось інший, точно не пам’ятаю, має скорочення ДСП. Сталася пригода, в доповідній вказали що провина в пригоді лежить на ДСП, на що отримали цілком резонну відповідь: «Всё понимаю, но причем тут фанера?».

А ще в часи коли міністром транспорту був Червоненко на одній селекторній нараді, де кожна з залізниць звітується що і як. Наради такі сліхають практично всі залізничники. Так от на одній з нарад з Одеси звітують що за якийсь там період завантажили скількись там тисяч полувагонів. Загоряється вогник з Києва і вгучномовець чути: «Почему полувагоны грузите? Грузите полные!». Для розуміння полувагон\footnote{Полувагон (рос.) Напіввагон — залізничний вантажний відкритий без даху вагон з високими бортами, призначений для перевезення навалювальних вантажів (руда, вугілля, флюси, лісоматеріали і т.п.), контейнерів і ін. Відкритий вагон має розвантажувальні люки в підлозі і торцеві стінки, що розкриваються, може мати глухий кузов.} це такий звичайний вантажний вагон без кришки.

А ще розказали про якусь програму чи щось що називаэться Соцваг чи якось так. Кажуть що це реєстр всіх вагонів всіх залізниць колишнього Радянського Союзу. І що тепер і після двадцяти з гаком  років незалежності інформація про вагони що належать львівській залізниці передається в Москву.

А ще в в кожному поїзді є чорна скринька — механічна біла стрічка що пише швидкість поїзда на кожній ділянці шляху, зупинки і так далі, і деколи машиністи вміють її пальчиком притримувати длятого щоб домогтися потрібних їм значень.

Також в тепловозах є якась така спеціальна кнопка, забув як називається що при деяких випадках вона має пищати і якщо машиність не натиснув йде екстерне гальмування. Зроблене то для того якщо раптом машиністу стало погано поїзд зупинився сам.




